\documentclass[fontsize=12pt, paper=a4, headinclude, twoside=false, parskip=half+, pagesize=auto, numbers=noenddot, open=right, toc=listof, toc=bibliography]{scrreprt}

%\usepackage[inner=4cm,outer=2cm]{geometry}
%\setlength{\oddsidemargin}{15,5pt}
%\setlength{\evensidemargin}{15,5pt}


%parskip:
  % full - Absätze haben großen Abstand
  % half - Absätze haben kleinen Abstand
  % off - Absätze haben Einzug (default)

% Bessere Unterstützung für PDF-Features
\usepackage[breaklinks=true]{hyperref}

%Schönere Schriftart laden
%\usepackage[latin1]{inputenc}
\usepackage[T1]{fontenc} % Ligaturen, richtige Umlaute im PDF
\usepackage[utf8]{inputenc}% UTF8-Kodierung für Umlaute usw
\usepackage[english]{babel} % Deutsche Silbentrennung verwenden
\usepackage{lmodern}
\renewcommand*\familydefault{\sfdefault}  %Zusatz für serifenlose Schrift.

%Zeilenabstand
\usepackage{setspace} % Zeilenabstand
\onehalfspacing % 1,5 Zeilen

% Schriften-Größen
\setkomafont{chapter}{\Huge\rmfamily} % Überschrift der Ebene
\setkomafont{section}{\Large\rmfamily}
\setkomafont{subsection}{\large\rmfamily}
\setkomafont{subsubsection}{\large\rmfamily}
\setkomafont{chapterentry}{\large\rmfamily} % Überschrift der Ebene in Inhaltsverzeichnis
\setkomafont{descriptionlabel}{\bfseries\rmfamily} % für description Umgebungen
\setkomafont{captionlabel}{\small\bfseries}
\setkomafont{caption}{\small}



% Einfachere Verwendung von korrekten Anführungszeichen
\usepackage[german=guillemets]{csquotes}
% oder german=quotes
% oder english=british oder english=american

%Mathematisches
\usepackage{amssymb}
\usepackage{amsmath}
\usepackage{amsthm}

%Quelltext einbinden
\usepackage{algorithm}
\usepackage{algorithmic}

%Abbildungen
\usepackage{graphicx}
\usepackage{caption}
\usepackage{subcaption}
\usepackage[verbose]{wrapfig}
\usepackage{float}
%\restylefloat{figure} %kannst du einen weiteren Positionierungsparameter [H] definieren. der setzt dir das bild an genau die stelle, wo du es haben willst. Ist allerdings auch nicht immer so praktisch.
% wenn du ein \pagebreak einfügst, gibt er dir vor der neuen seite noch alle gleitobjekte aus, die noch anstehen

%Zeichnen mit Tikz
\usepackage{tikz}
\usetikzlibrary{intersections,positioning,shapes.geometric,calc}

% Tabellen
\usepackage{multirow} % Tabellen-Zellen über mehrere Zeilen
\usepackage{multicol} % mehre Spalten auf eine Seite
\usepackage{tabularx} % Für Tabellen mit vorgegeben Größen
\usepackage{longtable} % Tabellen über mehrere Seiten
\usepackage{array}

%Bibliographie
\usepackage[square, comma, numbers, sort&compress, round]{natbib}
\usepackage{bibgerm} % Umlaute in BibTeX

%Umbenennung der vordefinierten definition- und example-Umgebung
\theoremstyle{definition}
\newtheorem{lecture}{Lecture}
\newtheorem{definition}{Definition}
\newtheorem{example}{Example}
\newtheorem{lemma}{Lemma}

% \newtheorem{theorem}{Satz}
% \newtheorem{constructing instructions}{Konstruktionsvorschrift}
% \newtheorem{properties}{Eigenschaften}
%\newtheorem{proposition}{Proposition}
%\newtheorem{korollar}{Corollary}
%\newtheorem{remark}{Remark}
%\newtheorem{consequences}{Consequences}
%\newtheorem{observation}{Observation}
%\newtheorem{conjecture}{Conjecture}
%\newtheorem{recall}{Recall}

\renewcommand{\labelenumi}{\roman{enumi})}

\renewcommand{\labelitemii}{$\bullet$}

\newcommand{\todo}[1]{
      {\colorbox{red}{ TODO: #1 }}
}
\newcommand{\todotext}[1]{
      {\color{red} TODO: #1} \normalfont
}

%bzgl `tocbasic` Warnung
\usepackage{scrhack}
 % Importiere die Einstellungen aus der Präambel
% hier beginnt der eigentliche Inhalt

\author{Lydia Buntrock}
\title{master thesis}
\date{August 2017}

\begin{document}
  % Titelseite
  \begin{titlepage}
    \pagestyle{empty}
  	
    	\vspace{20mm}
    	\begin{Large}
    	    \textbf{Origins and losses of parasitism}\\
          an analysis of the phylogenetic tree of life with a parsimony-like algorithm\\
    	\end{Large}

  	\clearpage
  \end{titlepage}

%---------------------------------------------------------------------------------------------------
%---------------------------------------------------------------------------------------------------
%--------------------------------------------------------------------------------------------------- 
\chapter*{Abstract}

\tableofcontents
\clearpage

%---------------------------------------------------------------------------------------------------
%---------------------------------------------------------------------------------------------------
%--------------------------------------------------------------------------------------------------- 
\chapter{Introduction}
  This paper is about the further development of parsimony algorithms for non-binary trees, applied 
  to the currently largest phylogeny synthesis tree of Open Tree Of Life, with the application to 
  the ancestral state reconstruction of parasitism. \\
  Researchers of the phylogenies have been dealt with the ancenstral state reconstruction in the 
  60s. The first methods were only brute force \todo{Quelle, siehe Fitch: Camin and Sokal 1965}. 
  Next came a set of parsimony algorithms such as: Fitch-parsimony \cite{Fitch1971}, 
  Wagner-parsimony \cite{Swofford1987} ... \todo{weitere?}. \\
  With more and more data, there is now the possibility to use more information to calculate the 
  probabilities of the ancestral states. In addition to the states of the leaves, algorithms could 
  also use branch lengths. The likelihood based algorithms came more in interest. \\
  Our focus came with another 'data extension'. We wanted to work with the biggest phylogenetic tree 
  that exists at this moment, which goes over all observed species. For most \todo{most?} species there is no 
  phylogeny, but only a taxonomic classification. So the biggest 'phylogenetic tree' is a synthesis 
  of phylogenetic trees filled with a taxonomic tree given by Open Tree of Life \cite{Hinchliff2015}.
  This tree is not binary and therefore the developed algorithms are not directly applicable. \\
  In this work, we have looked at the algorithms that are generally suited to our data, to develop 
  them further for the not binary case, and finally to compare their usability with our sythesis 
  tree. \\
  We have decided to consider only parsimony algorithms since we have no information on branch 
  lengths and no other additional information like different transition probabilities of our states.

%---------------------------------------------------------------------------------------------------
%---------------------------------------------------------------------------------------------------
%--------------------------------------------------------------------------------------------------- 
\chapter{Methods}
  \section{Simulation}
    \begin{itemize}
      \item build random binary trees, tag these (parameters: parasites vs free-living, beta-distribution)
      \item run fitch-parsimony, wagner-parsimony, our parsimony like algorithm
      \item build not binary tree (poisson distribution?)
      \item run new algorithms
      \item compare trees (distances)
    \end{itemize}
  \section{Implementation}
\chapter{Results}
\chapter{Discussion}

%-------------------------------------------------------------------------------------------------------------------------------------------------------------
%-------------------------------------------------------------------------------------------------------------------------------------------------------------
%-------------------------------------------------------------------------------------------------------------------------------------------------------------
\bibliography{bibliographie}
\bibliographystyle{alphadin}


\end{document}
