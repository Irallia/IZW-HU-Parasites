\documentclass[fontsize=12pt, paper=a4, headinclude, twoside=false, parskip=half+, pagesize=auto, numbers=noenddot, open=right, toc=listof, toc=bibliography]{scrreprt}

%\usepackage[inner=4cm,outer=2cm]{geometry}
%\setlength{\oddsidemargin}{15,5pt}
%\setlength{\evensidemargin}{15,5pt}


%parskip:
  % full - Absätze haben großen Abstand
  % half - Absätze haben kleinen Abstand
  % off - Absätze haben Einzug (default)

% Bessere Unterstützung für PDF-Features
\usepackage[breaklinks=true]{hyperref}

%Schönere Schriftart laden
%\usepackage[latin1]{inputenc}
\usepackage[T1]{fontenc} % Ligaturen, richtige Umlaute im PDF
\usepackage[utf8]{inputenc}% UTF8-Kodierung für Umlaute usw
\usepackage[english]{babel} % Deutsche Silbentrennung verwenden
\usepackage{lmodern}
\renewcommand*\familydefault{\sfdefault}  %Zusatz für serifenlose Schrift.
% \usepackage{MyriadPro}
% \renewcommand*\familydefault{\sfdefault}
% \usepackage{MinionPro}
% \renewcommand{\sfdefault}{Myriad-LF}

%Zeilenabstand
\usepackage{setspace} % Zeilenabstand
\onehalfspacing % 1,5 Zeilen

% Schriften-Größen
\setkomafont{chapter}{\Huge\rmfamily} % Überschrift der Ebene
\setkomafont{section}{\Large\rmfamily}
\setkomafont{subsection}{\large\rmfamily}
\setkomafont{subsubsection}{\large\rmfamily}
\setkomafont{chapterentry}{\large\rmfamily} % Überschrift der Ebene in Inhaltsverzeichnis
\setkomafont{descriptionlabel}{\bfseries\rmfamily} % für description Umgebungen
\setkomafont{captionlabel}{\small\bfseries}
\setkomafont{caption}{\small}

% Einfachere Verwendung von korrekten Anführungszeichen
\usepackage[german=guillemets]{csquotes}
% oder german=quotes
% oder english=british oder english=american

%Mathematisches
\usepackage{amssymb}
\usepackage{amsmath}
\usepackage{amsthm}

%Quelltext einbinden
\usepackage{algorithm}
\usepackage{algorithmic}

%Abbildungen
\usepackage{graphicx}
\usepackage{caption}
\usepackage{subcaption}
\usepackage[verbose]{wrapfig}
\usepackage{float}
%\restylefloat{figure} %kannst du einen weiteren Positionierungsparameter [H] definieren. der setzt dir das bild an genau die stelle, wo du es haben willst. Ist allerdings auch nicht immer so praktisch.
% wenn du ein \pagebreak einfügst, gibt er dir vor der neuen seite noch alle gleitobjekte aus, die noch anstehen

%Zeichnen mit Tikz
\usepackage{tikz}
\usetikzlibrary{intersections,positioning,shapes.geometric,calc}

% Tabellen
\usepackage{multirow} % Tabellen-Zellen über mehrere Zeilen
\usepackage{multicol} % mehre Spalten auf eine Seite
\usepackage{tabularx} % Für Tabellen mit vorgegeben Größen
\usepackage{longtable} % Tabellen über mehrere Seiten
\usepackage{array}
\usepackage{colortbl}
\newcommand\setrow[1]{\gdef\rowmac{#1}#1\ignorespaces}
\newcommand\clearrow{\global\let\rowmac\relax}

%Bibliographie
\usepackage[square, comma, numbers, sort&compress, round]{natbib}
\bibliographystyle{achemso}
% \usepackage{apalike}
% \bibliographystyle{apa}
% \usepackage{apacite}

% Default fixed font does not support bold face
\DeclareFixedFont{\ttb}{T1}{txtt}{bx}{n}{12} % for bold
\DeclareFixedFont{\ttm}{T1}{txtt}{m}{n}{12}  % for normal

% Custom colors
\usepackage{color}
\definecolor{deepblue}{rgb}{0,0,0.5}
\definecolor{deepred}{rgb}{0.6,0,0}
\definecolor{deepgreen}{rgb}{0,0.5,0}

% Code in a box
\usepackage{listings}
% Python style for highlighting
\lstset{
      language=Python,
      basicstyle=\ttm,
      otherkeywords={self},             % Add keywords here
      keywordstyle=\ttb\color{deepblue},
      emph={MyClass,__init__},          % Custom highlighting
      emphstyle=\ttb\color{deepred},    % Custom highlighting style
      stringstyle=\color{deepgreen},
      frame=tb,                         % Any extra options here
      showstringspaces=false            % 
}

%Umbenennung der vordefinierten definition- und example-Umgebung
% \theoremstyle{definition}
% \newtheorem{lecture}{Lecture}
% \newtheorem{definition}{Definition}
% \newtheorem{example}{Example}
% \newtheorem{lemma}{Lemma}
% \newtheorem{theorem}{Satz}
% \newtheorem{constructing instructions}{Konstruktionsvorschrift}
% \newtheorem{properties}{Eigenschaften}
%\newtheorem{proposition}{Proposition}
%\newtheorem{korollar}{Corollary}
%\newtheorem{remark}{Remark}
%\newtheorem{consequences}{Consequences}
%\newtheorem{observation}{Observation}
%\newtheorem{conjecture}{Conjecture}
%\newtheorem{recall}{Recall}

\renewcommand{\labelenumi}{\roman{enumi})}

\renewcommand{\labelitemii}{$\bullet$}

\newcommand{\todo}[1]{
      {\colorbox{red}{ TODO: #1 }}
}
\newcommand{\todotext}[1]{
      {\color{red} TODO: #1} \normalfont
}

\newcommand{\anmerkung}[1]{
      {\colorbox{yellow}{ Anmerkung: #1 }}
}
\newcommand{\anmerkungstext}[1]{
      {\color{orange} Anmerkung: #1} \normalfont
}

%bzgl `tocbasic` Warnung
\usepackage{scrhack}