\documentclass[fontsize=12pt, paper=a4, headinclude, twoside=false, parskip=half+, pagesize=auto, numbers=noenddot, open=right, toc=listof, toc=bibliography]{scrreprt}

%\usepackage[inner=4cm,outer=2cm]{geometry}
%\setlength{\oddsidemargin}{15,5pt}
%\setlength{\evensidemargin}{15,5pt}


%parskip:
  % full - Absätze haben großen Abstand
  % half - Absätze haben kleinen Abstand
  % off - Absätze haben Einzug (default)

% Bessere Unterstützung für PDF-Features
\usepackage[breaklinks=true]{hyperref}

%Schönere Schriftart laden
%\usepackage[latin1]{inputenc}
\usepackage[T1]{fontenc} % Ligaturen, richtige Umlaute im PDF
\usepackage[utf8]{inputenc}% UTF8-Kodierung für Umlaute usw
\usepackage[english]{babel} % Deutsche Silbentrennung verwenden
\usepackage{lmodern}
\renewcommand*\familydefault{\sfdefault}  %Zusatz für serifenlose Schrift.

%Zeilenabstand
\usepackage{setspace} % Zeilenabstand
\onehalfspacing % 1,5 Zeilen

% Schriften-Größen
\setkomafont{chapter}{\Huge\rmfamily} % Überschrift der Ebene
\setkomafont{section}{\Large\rmfamily}
\setkomafont{subsection}{\large\rmfamily}
\setkomafont{subsubsection}{\large\rmfamily}
\setkomafont{chapterentry}{\large\rmfamily} % Überschrift der Ebene in Inhaltsverzeichnis
\setkomafont{descriptionlabel}{\bfseries\rmfamily} % für description Umgebungen
\setkomafont{captionlabel}{\small\bfseries}
\setkomafont{caption}{\small}



% Einfachere Verwendung von korrekten Anführungszeichen
\usepackage[german=guillemets]{csquotes}
% oder german=quotes
% oder english=british oder english=american

%Mathematisches
\usepackage{amssymb}
\usepackage{amsmath}
\usepackage{amsthm}

%Quelltext einbinden
\usepackage{algorithm}
\usepackage{algorithmic}

%Abbildungen
\usepackage{graphicx}
\usepackage{caption}
\usepackage{subcaption}
\usepackage[verbose]{wrapfig}
\usepackage{float}
%\restylefloat{figure} %kannst du einen weiteren Positionierungsparameter [H] definieren. der setzt dir das bild an genau die stelle, wo du es haben willst. Ist allerdings auch nicht immer so praktisch.
% wenn du ein \pagebreak einfügst, gibt er dir vor der neuen seite noch alle gleitobjekte aus, die noch anstehen

%Zeichnen mit Tikz
\usepackage{tikz}
\usetikzlibrary{intersections,positioning,shapes.geometric,calc}

% Tabellen
\usepackage{multirow} % Tabellen-Zellen über mehrere Zeilen
\usepackage{multicol} % mehre Spalten auf eine Seite
\usepackage{tabularx} % Für Tabellen mit vorgegeben Größen
\usepackage{longtable} % Tabellen über mehrere Seiten
\usepackage{array}

%Bibliographie
\usepackage[square, comma, numbers, sort&compress, round]{natbib}
\usepackage{bibgerm} % Umlaute in BibTeX

%Umbenennung der vordefinierten definition- und example-Umgebung
\theoremstyle{definition}
\newtheorem{lecture}{Lecture}
\newtheorem{definition}{Definition}
\newtheorem{example}{Example}
\newtheorem{lemma}{Lemma}

% \newtheorem{theorem}{Satz}
% \newtheorem{constructing instructions}{Konstruktionsvorschrift}
% \newtheorem{properties}{Eigenschaften}
%\newtheorem{proposition}{Proposition}
%\newtheorem{korollar}{Corollary}
%\newtheorem{remark}{Remark}
%\newtheorem{consequences}{Consequences}
%\newtheorem{observation}{Observation}
%\newtheorem{conjecture}{Conjecture}
%\newtheorem{recall}{Recall}

\renewcommand{\labelenumi}{\roman{enumi})}

\renewcommand{\labelitemii}{$\bullet$}

\newcommand{\todo}[1]{
      {\colorbox{red}{ TODO: #1 }}
}
\newcommand{\todotext}[1]{
      {\color{red} TODO: #1} \normalfont
}

%bzgl `tocbasic` Warnung
\usepackage{scrhack}
 % Importiere die Einstellungen aus der Präambel
% hier beginnt der eigentliche Inhalt

\author{Lydia Buntrock}
\title{Masterarbeit}
\date{April 2014}

\begin{document}
  % Titelseite
  \begin{titlepage}
    \pagestyle{empty}
  	\begin{center}
      {\Large Freie Universität Berlin}\\
    	\begin{Huge}
      	Fachbereich Mathematik und Informatik\\
      	\vspace{3mm}
    	\end{Huge}
    	\vspace{20mm}
    	\begin{Large}
    	    \textbf{Origins of Parasitism}\\
          (Likelihood, parsimony and other algorithms for ancectral state reconstruction)\\
    	\end{Large}
    	\vspace{8mm}
      Version 1\\
      8.8.2017\\
    	\vspace{2cm}
    	Lydia Buntrock \\
      E-Mail: info@irallia.de\\
     	\vspace{5cm}
    	\textbf{Betreuer:}\\
      Prof. Dr. rer. nat. Emanuel Heitlinger\\
      \& \\
      Dr. Bernhard Y. Renard\\
  	\end{center}
  	\clearpage
  \end{titlepage}

  \tableofcontents
  \clearpage

  %-------------------------------------------------------------------------------------------------------------------------------------------------------------
  %-------------------------------------------------------------------------------------------------------------------------------------------------------------
  %-------------------------------------------------------------------------------------------------------------------------------------------------------------

  \chapter{Introduction}
    \begin{lecture}
      (21.4.17)
    \end{lecture}
    Related to computational systems biology\\
    Molecular biology
    \begin{itemize}
      \item Biological macromolecules
      \item Macromolecular interactions
      \item Pathways, networks, systems
    \end{itemize}
    (Subfields: pathway informatics, systems informatics)
    \begin{definition}
      \textbf{Systems biology:} Understand how components of a biological system interact to perform complex biological function.
    \end{definition}
    Challenges:
    \begin{itemize}
      \item Different levels of complexity
      \begin{itemize}
        \item Many components, huge amount of data, non-trivial interactions
      \end{itemize}
      \item Intuitive reasoning not sufficient
      \item Need for mathematical and computational models and tools
    \end{itemize}
    $\rightarrow$ Predictive biology\\
    \textbf{Research cycle}
    \todo{Research cycle...}\\

    Network topology
    \begin{itemize}
      \item Graph-based modeling
      \item Stoichiometric / constaint-based modeling
    \end{itemize}
    Network dynamics
    \begin{itemize}
      \item Continuous modeling
      \item Discrete modeling
      \item Stochastic modeling
      \item Hybrid modeling
    \end{itemize}

    \textbf{Important issues}
    \begin{itemize}
      \item Abstraction vs. precision
      \item Quantitative vs. qualitative
      \item Determinisitc vy. Non-deterministic
    \end{itemize}

    \textbf{Outline}
    \begin{enumerate}
      \item Continuous models
      \item Discrete models
      \item Constraint-based models
      \item Stochastic and hybrid models
    \end{enumerate}

  %-------------------------------------------------------------------------------------------------------------------------------------------------------------
  \chapter{Chemical kinetics}
    \begin{lecture}
      (21.4.17)
    \end{lecture}
    \section{Modeling simple reactions}
      $X, Y, \dots $ chemical species \\
      $x(t), y(t), \dots $ concentrations \\
      $\dot{x} = \dot{x}(t) = \frac{\mathrm d x}{\mathrm d t} (t)$ \\
      Modeling assumption: Reaction rate is propotional to the product of the reaction concentrations. \\
      Decay: $X -k-> \dot , \dot{x}=-kx$ (1) \\
      Transformation: $X -k-> Y, \dot{x}=-kx, \dot{y}=kx$ (2) \\
      Dissociation: $Z -k-> X+Y, \dot{z}=-kz, \dot{x}=kz, \dot{y}=kz$ \\
      Biomolecular reaction: $X+Y -k-> Z, \dot{z}=kxy, \dot{x}=-kxy = \dot{y}$ (z is a biolinear function, because it is linear in x and y) \\
      Reversible reaction: $X+Y <-k^-, k^+ -> Z, \dot{z}=k^+ xy-k^- z, \dot{x} = k^- z-k^+ xy = \dot{y}$ \\
      Dimerization: $X+X <-…-> Y [2X <-> Y]$ \\
      $\dot{x} = -2k^+ x^2 + 2k^- y$ \\
      $\dot{y} = -k^+ x^2 - k^- y$ \\
      *k stands for kinetic parameter \\




    \section{Chemical reaction networks}


    \begin{lecture}
      (24.4.17)
    \end{lecture}
    \begin{example}
      $n_S=3$ species: $S_1=H, S_2=0, S_3=H_2O$ \\
      $n_r=2$ reactions: $R_1: \underbrace{2H}_{\alpha_11} + \underbrace{O}_{\alpha_21} \longrightarrow \underbrace{H_2O}_{\beta_31}$
      $R_2: \underbrace{H_2O}_{\alpha_32} \underbrace{2H}_{\beta_12} + \underbrace{O}_{\beta_22}$
      $\Gamma = \bordermatrix{
        ~ & R_1 & R_2 \cr
        H & -2 & +2 \cr
        O & -1 & +1 \cr
        H_2O & +1 & -1 \cr}
      = \begin{pmatrix}
        k_1[H]^2[O] \\ k_2[H_2O]
      \end{pmatrix}$ \\
      $S(t) =
       \begin{pmatrix}
        S_1(t) \\ S_2(t) \\ S_3(t)
       \end{pmatrix}
       =
      \begin{pmatrix}
        [H] \\ [O] \\ [H_2O]
      \end{pmatrix}$ \\
      $R_1(S) = k_1[H]*[H]*[O] = [H]^2[O]$ \\
      $R_2(S) = k_2[H_2O]$ \\
      ...\\
      $\frac{\mathrm d S}{\mathrm d t} = \Gamma * R(S)$
      $\frac{\mathrm d [H]}{\mathrm d t} = -2k_1 [H]^2[O]+2k_2[H_2O]$ \\
      $\frac{\mathrm d [O]}{\mathrm d t} = -k_1 [H]^2[O]+k_2[H_2O]$ \\
      $\frac{\mathrm d [H_2O]}{\mathrm d t} = k_1 [H]^2[O]-k_2[H_2O]$ \\
    \end{example}

    bis Folie 2008...

    \begin{example}
      $\dot{x} = -kx, x(0)=1 (\dot{x}(t) = -kx(t))$ \\
      $n=1$: \\
      Assume $k=-2$ \todo{picture} \\
      $n=2$: \\
      $\dot{x}_1 = x_1+x_2$
      $\dot{x}_2 = x_1-x_2$
      $x(0)= (1 1)$
    \end{example}

    \begin{example}
      \todo{example via computation... see programm}
    \end{example}
    \begin{lecture}
      (24.4.17)
    \end{lecture}
    \subsection{Phase space}
      Autonomous equation $\dot{x}=f(x)$, with $x \in D \subseteq \mathbb{R}^n$. \\
      D is called \textbf{phase space}. \\
      $x(t) = (x1(t), \dots ,xn(t))$ is called \textbf{phase point}. \\
      When t varies, x(t) will move through phase space \textbf{trajectory} / \textbf{orbit} \\
      $f(x)$ can be interpreted as velocity vector. \\
      If the existence and uniqueness theorem applies, trajectories in phase space never intersect.
    \subsection{Steady states}
      \textbf{Nullcline}: $N_i = {x \in D | \dot{x}_i = f_i(x) = 0}$, for $i = 1, \dots , n$. \\
      A point $a \in D$ with $f(a) = 0$ (i.e., $f_i(a) = \dot{x}_i = 0, \forall i = 1, \dots ,n)$
      is called a \textbf{critical/singular/equilibrium point} or a \textbf{steady state}. \\
      It corresponds to the \textbf{equilibrium} or \textbf{stationary solution}
      $x(t) = a, \forall t$. \\
      It follows from the existence and uniqueness theorem that a steady state can never
      be reached from outside in finite time (otherwise two solutions would intersect).
  \subsection{Attractors and periodic solutions}
    A critical point $x = a$ of the equation $\dot{x} = f(x)$ is called a
    \textbf{positive attractor} if there exists a neighborhood
    $\omega a \subseteq \mathbb{R}^n$ of a such that
    $x(0) \in \Omega a implies lim_{t \rightarrow \infty} x(t) = a$. \\
    If this property holds for $t \rightarrow - \infty$, then $x = a$ is called a
    \textbf{negative attractor}.\\
    A solution $x(t)$ of $\dot{x} = f(x)$ is called \textbf{periodic} if there exists
    $T > 0$ such that $x(t + T) = x(t), \forall t \in \mathbb{R}$.
    \begin{lemma}
      Periodic solutions correspond to closed trajectories in phase space and vice versa.
    \end{lemma}
    A \textbf{limit cycle} is an isolated closed trajectory. \textbf{Isolated}
    means that neighboring trajectories are not closed; they spiral either toward
    or away from the limit cycle. \\
    \todo{picture 8}






\end{document}
