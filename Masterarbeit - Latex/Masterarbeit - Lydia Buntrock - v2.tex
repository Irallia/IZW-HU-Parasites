\documentclass[fontsize=12pt, paper=a4, headinclude, twoside=false, parskip=half+, pagesize=auto, numbers=noenddot, open=right, toc=listof, toc=bibliography]{scrreprt}

%\usepackage[inner=4cm,outer=2cm]{geometry}
%\setlength{\oddsidemargin}{15,5pt}
%\setlength{\evensidemargin}{15,5pt}


%parskip:
  % full - Absätze haben großen Abstand
  % half - Absätze haben kleinen Abstand
  % off - Absätze haben Einzug (default)

% Bessere Unterstützung für PDF-Features
\usepackage[breaklinks=true]{hyperref}

%Schönere Schriftart laden
%\usepackage[latin1]{inputenc}
\usepackage[T1]{fontenc} % Ligaturen, richtige Umlaute im PDF
\usepackage[utf8]{inputenc}% UTF8-Kodierung für Umlaute usw
\usepackage[english]{babel} % Deutsche Silbentrennung verwenden
\usepackage{lmodern}
\renewcommand*\familydefault{\sfdefault}  %Zusatz für serifenlose Schrift.

%Zeilenabstand
\usepackage{setspace} % Zeilenabstand
\onehalfspacing % 1,5 Zeilen

% Schriften-Größen
\setkomafont{chapter}{\Huge\rmfamily} % Überschrift der Ebene
\setkomafont{section}{\Large\rmfamily}
\setkomafont{subsection}{\large\rmfamily}
\setkomafont{subsubsection}{\large\rmfamily}
\setkomafont{chapterentry}{\large\rmfamily} % Überschrift der Ebene in Inhaltsverzeichnis
\setkomafont{descriptionlabel}{\bfseries\rmfamily} % für description Umgebungen
\setkomafont{captionlabel}{\small\bfseries}
\setkomafont{caption}{\small}



% Einfachere Verwendung von korrekten Anführungszeichen
\usepackage[german=guillemets]{csquotes}
% oder german=quotes
% oder english=british oder english=american

%Mathematisches
\usepackage{amssymb}
\usepackage{amsmath}
\usepackage{amsthm}

%Quelltext einbinden
\usepackage{algorithm}
\usepackage{algorithmic}

%Abbildungen
\usepackage{graphicx}
\usepackage{caption}
\usepackage{subcaption}
\usepackage[verbose]{wrapfig}
\usepackage{float}
%\restylefloat{figure} %kannst du einen weiteren Positionierungsparameter [H] definieren. der setzt dir das bild an genau die stelle, wo du es haben willst. Ist allerdings auch nicht immer so praktisch.
% wenn du ein \pagebreak einfügst, gibt er dir vor der neuen seite noch alle gleitobjekte aus, die noch anstehen

%Zeichnen mit Tikz
\usepackage{tikz}
\usetikzlibrary{intersections,positioning,shapes.geometric,calc}

% Tabellen
\usepackage{multirow} % Tabellen-Zellen über mehrere Zeilen
\usepackage{multicol} % mehre Spalten auf eine Seite
\usepackage{tabularx} % Für Tabellen mit vorgegeben Größen
\usepackage{longtable} % Tabellen über mehrere Seiten
\usepackage{array}

%Bibliographie
\usepackage[square, comma, numbers, sort&compress, round]{natbib}
\usepackage{bibgerm} % Umlaute in BibTeX

%Umbenennung der vordefinierten definition- und example-Umgebung
\theoremstyle{definition}
\newtheorem{lecture}{Lecture}
\newtheorem{definition}{Definition}
\newtheorem{example}{Example}
\newtheorem{lemma}{Lemma}

% \newtheorem{theorem}{Satz}
% \newtheorem{constructing instructions}{Konstruktionsvorschrift}
% \newtheorem{properties}{Eigenschaften}
%\newtheorem{proposition}{Proposition}
%\newtheorem{korollar}{Corollary}
%\newtheorem{remark}{Remark}
%\newtheorem{consequences}{Consequences}
%\newtheorem{observation}{Observation}
%\newtheorem{conjecture}{Conjecture}
%\newtheorem{recall}{Recall}

\renewcommand{\labelenumi}{\roman{enumi})}

\renewcommand{\labelitemii}{$\bullet$}

\newcommand{\todo}[1]{
      {\colorbox{red}{ TODO: #1 }}
}
\newcommand{\todotext}[1]{
      {\color{red} TODO: #1} \normalfont
}

%bzgl `tocbasic` Warnung
\usepackage{scrhack}
 % Importiere die Einstellungen aus der Präambel
% hier beginnt der eigentliche Inhalt

\author{Lydia Buntrock}
\title{master thesis}
\date{August 2017}

\begin{document}
  % Titelseite
  \begin{titlepage}
    \pagestyle{empty}
  	\begin{center}
      {\Large Freie Universität Berlin}\\
    	\begin{Huge}
      	Fachbereich Mathematik und Informatik\\
      	\vspace{3mm}
    	\end{Huge}
    	\vspace{20mm}
    	\begin{Large}
    	    \textbf{Origins and losses of parasitism}\\
          an analysis of the phylogenetic tree of life with a parsimony-like algorithm\\
    	\end{Large}
    	\vspace{8mm}
      Version 2\\
      2.10.2017\\
    	\vspace{2cm}
    	Lydia Buntrock \\
      E-Mail: info@irallia.de\\
     	\vspace{5cm}
      \textbf{Betreuer:}\\
      Dr. Bernhard Y. Renard\\
      \& \\
      Prof. Dr. rer. nat. Emanuel Heitlinger\\      
  	\end{center}
  	\clearpage
  \end{titlepage}

%---------------------------------------------------------------------------------------------------
%---------------------------------------------------------------------------------------------------
%--------------------------------------------------------------------------------------------------- 
\chapter*{Abstract}
  \todotext{The abstract} is a concise and accurate summary of the scholarly work described in the 
  document. It states the problem, the methods of investigation, and the general conclusions, and 
  should not contain tables, graphs, complex equations, or illustrations. There is a single 
  scholarly abstract for the entire work, and it must not exceed 350 words in length.

%---------------------------------------------------------------------------------------------------
%---------------------------------------------------------------------------------------------------
%--------------------------------------------------------------------------------------------------- 
\chapter*{Summary}
  \todotext{1 page} presenting the research problem, the main results, conclusion and how the thesis 
  advances the field. \\
  The lay or public summary explains the key goals and contributions of the research/scholarly work 
  in terms that can be understood by the general public. It must not exceed 150 words in length.

%---------------------------------------------------------------------------------------------------
%---------------------------------------------------------------------------------------------------
%--------------------------------------------------------------------------------------------------- 
\chapter*{Preface}
  \todotext{The Preface} must include a statement indicating the student's contribution to the 
    following:
  \begin{itemize}
    \item Identification and design of the research program,
    \item Performance of the various parts of the research, and
    \item Analysis of the research data.
  \end{itemize}
  Certain additional elements may also be required, as specified below.
  \begin{itemize}
    \item If any of the work presented in the thesis has led to any publications or submissions, all 
      of these must be listed in the Preface. Bibliographic details should include the title of the 
      article and the name of the publisher (if the article has been accepted or published), and the 
      chapter(s) of the thesis in which the associated work is located.
    \item If the work includes publications or material submitted for publication, the statement 
      described above must detail the relative contributions of all collaborators and co-authors 
      (including supervisors and members of the supervisory committee) and state the proportion of 
      research and writing conducted by the student. For further details, see “Including Published 
      Material in a Thesis or Dissertation”.
    \item If the work includes other scholarly artifacts (such as film and other audio, visual, and 
      graphic representations, and application-oriented documents such as policy briefs, curricula, 
      business plans, computer and web tools, pages, and applications, etc.), all of these must be 
      listed in the Preface (with bibliographical information, if applicable).
    \item If ethics approval was required for the research, the Preface must name the responsible 
      UBC Research Ethics Board, and report the project title(s) and the Certificate Number(s) of 
      the Ethics Certificate(s) applicable to the project.
  \end{itemize}
  In a thesis where the research was not subject to ethics review, produced no publications, and was 
    designed, carried out, and analyzed by the student alone, the text of the Preface may be very 
    brief. Samples are available on this website and in the University Library's online repository 
    of accepted theses. \\
  The content of the Preface must be verified by the student's supervisor, whose endorsement must 
    appear on the final Thesis/Dissertation Approval form. \\
  Acknowledgements, introductory material, and a list of publications do not belong in the Preface. 
    Please put them respectively in the Acknowledgements section, the first section of the thesis, 
    and the appendices. \\

  A preface is best understood, I believe, as standing outside the book proper and being about the 
    book. In a preface an author explains briefly why they wrote the book, or how they came to write 
    it. They also often use the preface to establish their credibility, indicating their experience 
    in the topic or their professional suitability to address such a topic. Sometimes they 
    acknowledge those who inspired them or helped them (though these are often put into a separate 
    Acknowledgments section). Using an old term from the study of rhetoric, a preface is in a sense 
    an “apology”: an explanation or defense.

\tableofcontents
\clearpage

6. LIST OF TABLES (REQUIRED IF DOCUMENT HAS TABLES) \\
7. LIST OF FIGURES (REQUIRED IF DOCUMENT HAS FIGURES) \\
8. LIST OF ILLUSTRATIONS (REQUIRED IF DOCUMENT HAS ILLUSTRATIONS) \\
9. LISTS OF SYMBOLS, ABBREVIATIONS OR OTHER (ADVISABLE IF APPLICABLE) \\
10. GLOSSARY (OPTIONAL) \\
11. ACKNOWLEDGEMENTS (OPTIONAL) \\
  Students may include a brief statement acknowledging the contribution to their research and 
    studies from various sources, including (but not limited  to) \\

  Their research supervisor and committee, \\
  Funding agencies, \\
  Professional or community collaborators, \\
  Fellow students, and \\
  Family and friends. \\

12. DEDICATION (OPTIONAL) (Widmung) \\

13. DOCUMENT BODY \\
  The text of the thesis must contain the following elements, presented to conform to the standards 
    and expectations of the relevant academic discipline. In some cases, the ordering of these 
    ingredients may differ from the one shown here. \\

%---------------------------------------------------------------------------------------------------
%---------------------------------------------------------------------------------------------------
%--------------------------------------------------------------------------------------------------- 
\chapter{Introduction}
  \todotext{A. Introduction.} The thesis must clearly state its theme, hypotheses and/or goals
    (sometimes called “the research question(s)”), and provide sufficient background information to 
    enable a non-specialist scholar to understand them. It must contain a thorough review of 
    relevant literature, perhaps in a separate chapter. \\

  On the first page, you should present
  \begin{itemize}
    \item The area of research (e.g. implementation of information systems)
    \item The most relevant previous findings in this area
    \item Your research problem and why this is worthwhile studying
    \item The objective of the thesis: how far you hope to advance knowledge in the field
  \end{itemize}
  Target group \\
  To whom are you writing, what do you assume that the reader know? A normal target groupd would be 
    new Master students. \\
  Personal motivation \\
  Why did you choose this topic? \\
  Research method in brief \\
  How will you find out? \\
  Structure of the report \\
  A paragraph about each chapter. What is the main contribution of the chapter? How do they relate? \\

  %--------------------------------------------------------------------------------------------------- 
  \section{Definitions}
    \begin{definition} parsimony \cite{Cunningham1998}
    \end{definition}
    \begin{definition} taxonomy \\
      Taxonomic tree – Taxonomy is the classification, identification and naming of organisms. It is 
        usually richly informed by phylogenetics, but remains a methodologically and logically 
        distinct discipline. The Taxonomic levels are Kingdom, Phylum, Class, Order, Family, Genus, 
        Species and maybe some superclasses between them. \\
      coarser classification \\
      In each level the nodes can have more than 2 children
    \end{definition}
    \begin{definition} phylogeny \\
      Phylogenetic tree – Phylogenetics is the study of evolutionary history and the relationships 
        among individuals or groups of organisms. The result ot these analysis is a  phylogeny or 
        phylogenetic tree. The phylogenetic tree is a hypothesis about the history of the evolutionary 
        relationships of a group of organisms. \\
      A perfect tree would be binary \\
      no levels only time
      \todo{synthesis tree} \cite{Hinchliff2015}
    \end{definition}
    \begin{definition} parasite \\
      Parasitism – Parasitism is an interaction relationship between two organisms living together in 
        more or less intimate association in a relationship in which association is disadvantageous or 
        destructive to one of the organisms. \\
      http://www.ebi.ac.uk/QuickGO/term/GO:0072519 \\
      ontology definition of the interaction 'is parasite of' from GloBI \cite{Poelen2014} the database we use here. \\
      exist very different types of parasites: endoparasites, broodparasitism, ... \\
      40\% of species are parasites, but the most of them are understudied. \cite{Windsor1998}
      
      ontology definition.. explanation... \\
      Property Hierarchy from ontobee: \\
      ecologically related to $\supseteq$ biotically interacts with $\supseteq$ participates in a 
        biotic-biotic interaction with $\supseteq$ symbiotically interacts with $\supseteq$ has host 
        $\supseteq$ hyperparasite of, pathogen of, parasite of $\supseteq$ parasitoid of, obligate 
        parasite of, facultative parasite of, stem parasite of, root parasite of, hemiparasite of, 
        lays eggs in, ectoparasite of, endoparasite of, mesoparasite of, kleptoparasite of
    \end{definition}

  %--------------------------------------------------------------------------------------------------- 
  \section{Motivation}
    Weinstein und Kuris Paper \cite{Weinstein2016} \\
    study parasites:
    \begin{itemize}
      \item für jedes Ökosystem unentbehrlich [Marius]
      \item Wechselwirkungen in 75\% der Verbindungen in den Nahrungsmittelbahnen [Marius]
      \item agglomerieren Schadstoffe, binden 30 bis 50\% der Schadstoffmasse innerhalb eines Ökosystem [Marius]
      \item Dezimierung der Bevölkerung der störenden Arten [Marius]
    \end{itemize}
    big data:
    \begin{itemize}
      \item data accumulates faster than ever. Biological data is no exception but scientists still struggle to harvest this rapidly growing pile of databases. [Marius]      
    \end{itemize}
    origins of parasitism: \\
    Many studies tried to find the origins of parasitism. For example in Nematodes [14, 29], in Lice 
      [28], in Plathylminthes [30, 31] and in Metazoa in general [32, 33, 1] scientists searched for 
      these origins. The most recent analysis is the one from Sara Weinstein and Armand M. Kuris. 
      They came to the conclusion that parasitism evolved at least 223 times indepedently in the 
      kingdom of Animalia. They found most of the origins inside the phylum Arthropoda [1]. [Marius] \\
    Wenn wir die Ursprünge des Parasitums verstehen können, können wir die Evolution von Pathogenen 
      vorhersagen und welche genetischen Anpassungen für eine Transition vom freien Leben zu einem 
      parasitären Lebensstil notwendig sind [21]. Es könnte sogar zu einem Punkt führen, an dem wir 
      die Immunität gegen Antibiotika von Pathogenen vorhersagen konnten. [Marius]


%---------------------------------------------------------------------------------------------------
%---------------------------------------------------------------------------------------------------
%--------------------------------------------------------------------------------------------------- 
\chapter{Research/Scholarship}
  The account of the scholarly work should be presented in a 
    manner suitable for the field. It should be complete, systematic, and sufficiently detailed to 
    enable a reader to understand how the data were gathered and analyzed, and how to apply similar 
    methods in another study. Notation and formatting must be consistent throughout the thesis, 
    including units of measure, abbreviations, and the numbering scheme for tables, figures, 
    footnotes, and citations. One or more chapters may consist of material published (or submitted
    for publication) elsewhere, or other artifacts (e.g., film, application-oriented documents) 
    placed in a scholarly context. See “Including Published Material in a Thesis or Dissertation” 
    for additional details. \\

  %---------------------------------------------------------------------------------------------------
  %--------------------------------------------------------------------------------------------------- 
  \section{Related literature and theoretical focus}
    A survey of the literature (journals, conferences, book chapters) on the areas that are relevant 
      to your research question. One section per area. \\
    The chapter should conclude with a summary of the previous research results that you want to 
      develop further or challenge. The summary could be presented in a model, a list of issues, etc. 
      Each issue could be a chapter in the presentation of results. They should definitely be 
      discussed in the discussion / conclusion of the thesis. 

  %---------------------------------------------------------------------------------------------------
  %--------------------------------------------------------------------------------------------------- 
  \section{Presentation of the part of the world to be studied}
    Which could be the health system in Mozambique, the rural areas where the telecentres are located, 
      the financial businesses in which the accounting system you will develop will be used. 

  %---------------------------------------------------------------------------------------------------
  %--------------------------------------------------------------------------------------------------- 
  \section{Method}
    The research method by which you will investigate the world.
    \begin{itemize} 
      \item A short summary of the available methods
      \item Your choice
      \item Detailed report of how you actually carried out your research. Presenting how you selected 
        the people taking part is of special importance.
    \end{itemize}
  
  %---------------------------------------------------------------------------------------------------
  %--------------------------------------------------------------------------------------------------- 
  \section{... Your research results (and discussion)}
    Which could be the system you made and the reasons for various design decisions, what your 
      interview objects said, observations of people using a computer system, stories of a development 
      process, numeral data from a questionnaire, etc. \\
    The discussion of the findings can be included in these chapters, or the discussion can be put in 
      a separate chapter.  \\
    The issues from the theory chapter 2 should be discussed here.

%---------------------------------------------------------------------------------------------------
%---------------------------------------------------------------------------------------------------
%--------------------------------------------------------------------------------------------------- 
\chapter{Conclusion}
  In this section the student must demonstrate his/her mastery of the field and 
    describe the work's overall contribution to the broader discipline in context. A strong 
    conclusion includes the following:
    \begin{itemize}
      \item Conclusions regarding the goals or hypotheses presented in the Introduction,
      \item Reflective analysis of the scholarly work and its conclusions in light of current 
        knowledge in the field,
      \item Comments on the significance and contribution of the scholarship reported,
      \item Comments on strengths and limitations of the research/scholarship,
      \item Discussion of any potential applications of the findings, and
      \item A description of possible future research directions, drawing on the work reported.
    \end{itemize}
    A submission's success in addressing the expectations above is appropriately judged by experts 
      in the relevant discipline. Students should rely on their research supervisors and committee 
      members for guidance. Doctoral students should also take into account the expectations 
      articulated in the University's “Instructions for Preparing the External Examiner's Report”.

  (Summary of the problem, the main findings and the discussion. Structured according to the issues 
    in chapter 2. \\
  Comparison with the literature presented in chapter 2: how do your results fill in, advance or 
    contradict previously reported research? \\
  What are the implications of your research for people working in the field that you have studies?)

%-------------------------------------------------------------------------------------------------------------------------------------------------------------
%-------------------------------------------------------------------------------------------------------------------------------------------------------------
%-------------------------------------------------------------------------------------------------------------------------------------------------------------
\bibliography{bibliographie}
\bibliographystyle{alphadin}

%---------------------------------------------------------------------------------------------------
%---------------------------------------------------------------------------------------------------
%--------------------------------------------------------------------------------------------------- 
\chapter*{Eigenständigkeitserklärung}
  Hiermit versichere ich, dass ich die vorliegende Masterarbeit selbständig verfasst habe. \\
  Ich versichere, dass ich keine anderen als die angegebenen Quellen benutzt und alle wörtlich oder sinngemäß aus anderen Werken übernommenen Aussagen als solche gekennzeichnet habe, und dass die eingereichte Arbeit weder vollständig noch in wesentlichen Teilen Gegenstand eines anderen Prüfungsverfahren gewesen ist. \\
  \vspace{2 cm} \\
  (Datum und Unterschrift)
  \todo{englisch!}

%---------------------------------------------------------------------------------------------------
%---------------------------------------------------------------------------------------------------
%--------------------------------------------------------------------------------------------------- 
\chapter{APPENDICES}
  \todotext{Appendices} must be limited to supporting material genuinely subsidiary to the main 
    argument of the work. They must only include material that is referred to in the document.\\
  Material suitable for inclusion in appendices includes the following:
  \begin{itemize} 
    \item Additional details of methodology and/or data
    \item Diagrams of specialized equipment developed
    \item Copies of questionnaires or surveys used in the research
    \item Scholarly artifacts (e.g., film and other audio, visual, and graphic representations, and 
      application-oriented documents such as policy briefs, curricula, business plans, computer and 
      web applications, etc.) not included in the body of the thesis
  \end{itemize}
  Do not include copies of the Ethics Certificates in the Appendices. \\
  Material supplemental to the thesis but not appropriate to include in the appendices (e.g., raw 
    data, original plan for research and analyses) can be archived in cIRcle as Supplementary 
    Materials.


\end{document}
